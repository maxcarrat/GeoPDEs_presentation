\newcommand{\hpd}[0]{\mbox{$hp$-$d$}\xspace}

\newcommand{\refSection}[1]{Section~\ref{#1}}
\newcommand{\refAlgorithm}[1]{Algorithm~\ref{#1}}
\newcommand{\refFigure}[1]{Figure~\ref{#1}}
\newcommand{\refTable}[1]{Table~\ref{#1}}
\newcommand{\refEquation}[1]{(\ref{#1})}

\newcommand{\redPart}[1]{\color{red}#1\color{black}}

\newcommand{\etal}{~\textit{et al.}\xspace}

\newcommand{\tensor}[1]{\boldsymbol{{#1}}}
\newcommand{\trans}{^{\top}}
\newcommand{\integral}[3]{\int\limits_{#1} #2 \, \mathrm{d}#3}

\newcolumntype{C}[1]{>{\centering\arraybackslash}m{#1}}
\newcolumntype{R}[1]{>{\raggedright\arraybackslash}m{#1}}
\newcolumntype{L}[1]{>{\raggedleft\arraybackslash}m{#1}}

\newcommand\restr[2]{{% we make the whole thing an ordinary symbol
		\left.\kern-\nulldelimiterspace % automatically resize the bar with \right
		#1 % the function
		%\vphantom{\big|} % pretend it's a little taller at normal size
		\right|_{#2} % this is the delimiter
}}
%\newcommand{\revOne}[1]{#1\xspace}
%\newcommand{\revTwo}[1]{#1\xspace}
%\newcommand{\revBoth}[1]{#1\xspace}
%\newcommand{\delete}[1]{\xspace}

\newcommand{\includeTikzFile}[3]{
  file = #1\\
  width = #2

  \renewcommand{\figFilename}{#1}
  \tikzpicturedependsonfile{\figFilename.tikz}
  \tikzsetnextfilename{\figFilename}
  \includegraphics[width = #2, 
                   height = #3]
                   {\figFilename.tikz}%
}

\newcommand{\Bezier}{B\'ezier}
\newcommand\compactDots{\ifmmode\ldots\else\makebox[10cm][c]{.\hfil.\hfil.}\fi}
\newcommand{\onTop}[2]{\stackrel{\mathclap{#1}}{#2}}

\newcommand{\supp}[1]{\textnormal{supp}(#1)}
\newcommand{\glob}{glob}
\newcommand{\loc}{loc}

\makeatletter
\def\namedlabel#1#2{\begingroup
	\def\@currentlabel{#2}%
	\label{#1}\endgroup
}
\makeatother

\makeatletter
\renewcommand{\todo}[2][]{\tikzexternaldisable\@todo[#1]{#2}\tikzexternalenable}
\makeatother

\usepackage{etoolbox}
\AtBeginEnvironment{mysmallmatrix}{%
\setstretch{0.9}%
\setlength{\arraycolsep}{2pt}%
}%

\renewcommand\arraystretch{1.2}
\newenvironment{mysmallmatrix}%
{%
	%\arraycolsep=3pt%
%	\small%
	\begin{matrix}%
}%
	{\end{matrix}}%

\newenvironment{myverysmallmatrix}%
{%
	%\arraycolsep=3pt%
	%	\small%
	\begin{matrix}%
	}%
	{\end{matrix}}%
\AtBeginEnvironment{myverysmallmatrix}{%
	\footnotesize%
	\setstretch{0.55}%
	\setlength{\arraycolsep}{1.5pt}%
}%

\newcommand{\overbar}[1]{\mkern 0.1mu\overline{\mkern-0.1mu#1\mkern-0.1mu}\mkern 0.1mu}
\setcounter{MaxMatrixCols}{30}

%\newtheorem{definition}{Definition}[section]

\makeatletter
\def\bbordermatrix#1{\begingroup \m@th
	\@tempdima 4.75\p@
	\setbox\z@\vbox{%
		\def\cr{\crcr\noalign{\kern2\p@\global\let\cr\endline}}%
		\ialign{$##$\hfil\kern2\p@\kern\@tempdima&\thinspace\hfil$##$\hfil
			&&\quad\hfil$##$\hfil\crcr
			\omit\strut\hfil\crcr\noalign{\kern-\baselineskip}%
			#1\crcr\omit\strut\cr}}%
	\setbox\tw@\vbox{\unvcopy\z@\global\setbox\@ne\lastbox}%
	\setbox\tw@\hbox{\unhbox\@ne\unskip\global\setbox\@ne\lastbox}%
	\setbox\tw@\hbox{$\kern\wd\@ne\kern-\@tempdima\left[\kern-\wd\@ne
		\global\setbox\@ne\vbox{\box\@ne\kern2\p@}%
		\vcenter{\kern-\ht\@ne\unvbox\z@\kern-\baselineskip}\,\right]$}%
	\null\;\vbox{\kern\ht\@ne\box\tw@}\endgroup}
\makeatother

\tikzset{
	invisible/.style={opacity=0},
	visible on/.style={alt={#1{}{invisible}}},
	alt/.code args={<#1>#2#3}{%
		\alt<#1>{\pgfkeysalso{#2}}{\pgfkeysalso{#3}} % \pgfkeysalso doesn't change the path
	},
}

\newcommand{\mytikzmark}[2]{%
	\tikz[overlay, remember picture, anchor=base, baseline=(#1.base), font=\tiny]%
	\node (#1) {#2};%
	%\tikzmark{#1}#2%
}%

\tikzset{%
	highlight0/.style={rectangle,fill=black!15,draw,
		fill opacity=1,draw opacity=0.2,inner sep=-2.2pt}
}%
\tikzset{%
	highlight1/.style={rectangle,fill=blue!15,draw,
		fill opacity=1,draw opacity=0.2,thin,inner sep=-2.2pt}
}%
\tikzset{%
	highlight2/.style={rectangle,fill=orange!15,draw,
		fill opacity=1,draw opacity=0.2,thin,inner sep=-2.2pt}
}%

\newcommand{\Highlight}[4][submatrix]{%
	%\tikz[overlay, remember picture]{
	\node[#2,fit=(#3.north west) (#4.south east)] (#1) {};%
	%}
}%
\newcommand{\AdjustRight}[2]{%
	%\tikz[overlay, remember picture, inner sep=0pt]{%
	\node[overlay, remember picture, inner sep=0pt] (#1adj) at (#2.east |- #1.south) {};%
	%}%
}%
\newcommand{\AdjustLeft}[2]{%
	%\tikz[overlay, remember picture, inner sep=0pt]{%
	\node[overlay, remember picture, inner sep=0pt] (#1adj) at (#2.west |- #1.north) {};%
	%}%
}%
